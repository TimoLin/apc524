\documentclass[11pt]{article}
\usepackage{fullpage}
\usepackage{array}
\usepackage{graphicx}
\usepackage{amssymb, amsmath}
\usepackage{hyperref}
\begin{document}


\title{\textbf{Tabulation of Chemical Source Terms for Turbulent
    Combustion Simulations}}

\author{Emmet Cleary: emcleary@princeton.edu \and Daniel Floryan:
  dfloryan@princeton.edu \and Jeffry Lew: jklew@princeton.edu \and
  Bruce Perry: baperry@princeton.edu \and Emre Turkoz:
  eturkoz@princeton.edu} 

\date{21 November 2014 }
\maketitle

\section{Introduction}

Turbulent combustion simulations require closure of a chemical source
term. This is not trivial, as it follows highly nonlinear Arrhenius
kinetics and can depend on numerous stiffly-coupled chemical
reactions. One approach, rather than evaluating these terms on the
fly, is to calculate a set thermochemical states \textit{a priori} and
to use them when solving conservation equations. When combined with
flamelet models\footnote{Pierce \textit{et al.}, J. Fluid Mech. (2004)
  vol. 504, pp. 73-97.}, chemical source term tabulation greatly
facilitates Large Eddy Simulations (LES) of turbulent reacting flows.

The challenge with these tabulation methods is knowing how to identify
each tabulated term as needed.  This is done by tabulating source
terms against a predetermined variable. The trick is to identify a
single variable that uniquely identifies each thermochemical
state. Temperature is the most obvious one: the more a reaction
proceeds, the more heat is released. However, filtering conservations
equations of energy for LES leads to a closure problem. It is far
easier to filter species conservation equations, but a single chemical
species is usually insufficient to identify the thermochemical state
uniquely. Rather, one must take linear combinations of several
species, called a progress variable, to define a mass-based
conservation equation that is suitable for turbulent combustion
simulations. Once a progress variable is chosen, the thermochemical
states can be sorted, convoluted by a probability density function
(PDF), and interpolated as needed to generate a table for chemical
source terms.

\section{Project Summary}

Our project will build off outputs from existing research codes
(FlameMaster) to facilitate and even automate the selection of
progress variable. The codes will also sort chemical source terms into
a table through a variety of interpolation schemes, integration
schemes, and PDFs. 

% Insert flow charts here along with descriptions.


% Presumably this paragraph can be removed.
FlameMaster models the chemistry, outputting columns of data
(temperatures, species mass fractions, source terms, and more) which
define a thermochemical state. NGA takes these inputs, arranges them
into a table, and uses the table during simulations. Currently no code
exists to identify suitable progress variables, which are instead
chosen arbitrarily or by (painfully) plotting many progress
variables. The tabulation codes in NGA must be rewritten and/or
modified every time a table's dimensionality changes, for different
combustion modes (premixed or non-premixed), etc.


\section{Architecture}
Our project consists of two related parts: choosing the progress
variable, and generating the tables. We will write different parts of
our code in Python and C++.



\subsection{Python Interface}




\subsection{C++ Interface}



\section{Milestones}
We plan to complete this project by accomplishing incremental milestones. A week-by-week breakdown of deliverables is shown below:

\begin{itemize}
\item \textbf{November 17-21, 2014:} Complete design document
\item \textbf{November 24-28, 2014:}
  \begin{itemize}
  \item
  \end{itemize}
\item \textbf{December 1-5, 2014:}
  \begin{itemize}
  \item Present at design review (overview of project & discuss specifics of design)
  \item Prototype ready (shows approximately what we are trying to do and what our system will look like)
  \item 
  \end{itemize}
\item \textbf{December 8-12, 2014:} Protoype
  \begin{itemize}
  \item Alpha version ready (almost working version of core functionality)
  \item
  \end{itemize}
\item \textbf{December 15-19, 2014:}
  \begin{itemize}
  \item 
  \end{itemize}
\item \textbf{December 22-26, 2014:}
  \begin{itemize}
  \item
  \end{itemize}
\item \textbf{December 29, 2014 - January 2, 2015:}
  \begin{itemize}
  \item
  \end{itemize}
\item \textbf{January 5-9, 2015:}
  \begin{itemize}
  \item Project presentation
  \item
  \end{itemize}
\item \textbf{January 12-15, 2015:}
  \begin{itemize}
  \item Final submission (code with automated tests, software manual, and report)
  \end{itemize}
\end{itemize}



\section{Risks and Open Issues}

% Risks
% -- handling cases where monotonic progress variables are not possible
% -- ????




\end{document}

 
